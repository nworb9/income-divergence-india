\documentclass[a4paper, 11pt]{article}

%% Language and font encodings
\usepackage[english]{babel}
\usepackage[utf8x]{inputenc}
\usepackage[T1]{fontenc}
\usepackage{mwe}
\usepackage[justification=centering]{caption}


%% Sets page size and margins
\usepackage[a4paper,top=2.5cm,bottom=2cm,left=2.7cm,right=2.7cm,marginparwidth=1.75cm]{geometry}

%% Useful packages
\usepackage{amsmath}
\usepackage{graphicx}
\usepackage[colorinlistoftodos]{todonotes}
\usepackage[round,sort,semicolon]{natbib}
\usepackage{bibentry}
\usepackage{csquotes}
\usepackage{tfrupee}
\usepackage{dcolumn}
\usepackage{setspace}
\spacing{1.5}
\usepackage{tabularx}
\usepackage[colorlinks=true, allcolors=blue]{hyperref}
\usepackage{url}
  \let\oldurl\url
\usepackage[colorlinks=true]{hyperref}
  \let\linkurl\url
  \let\url\oldurl
\usepackage[export]{adjustbox}

\setlength{\parindent}{4em}
\setlength{\parskip}{1em}

\title{Income Inequality in India: A Disaggregated View}
\author{B041315\\Dr. Sean Brocklebank\thanks{Special thanks to Sean for answering every question I had, no matter how trivial.  Thank you also to Dr. Liang Bai, Dr. Shaun Bevan, Johannes Eigner and Dr. Nicholas Meyers for their econometrics help, Dr. Andreas Steinhauer and Stuart Baumann for their help with RStudio, and Szymon Sacher for teaching me LaTeX.}}

\begin{document}


\bibliographystyle{apalike}
\begin{titlepage}
\maketitle

\begin{abstract}
\textit{This paper uses panel data analysis to examine the evolution of income growth and apparent increase in inequality across 15 Indian states during the post-reform period from 1991-2011.  While there was a lack of support for unconditional income convergence, strong evidence of conditional convergence to a stratified distribution of steady-states is found when differences in population growth, public investment, infrastructure, financial infrastructure, health, education and sectoral composition are controlled for.  The findings suggest that there remains a role for government in promoting regionally balanced development across Indian states, especially in promoting private investment and resolving productivity deficiencies in the poorest states.}
\end{abstract}
\end{titlepage}

\section{Introduction}

In 1991, India underwent extensive structural reform measures in exchange for a loan from the International Monetary Fund (IMF).  The sharp transition from a mixed economy to a market one should have produced significant efficiency gains.  Specifically, neoclassical theory predicts that the resulting growth should not only be shared among India's states, but that the faster growth of poorer states relative to rich ones would cause state incomes to converge on each other.  However, a number of features of India's political economy and the general inadequacy of neoclassical growth theory acted against this outcome.  Following liberalization, India's per capita income skyrocketed from \$375 in 1990 to \$1,456 in 2013, but its Gini coefficient rose from 45 to 51 in the same period \citep{jain2016sharing}.  The rise in inequality overshadowed the achievements in poverty reduction, and even these statistics hide a more nuanced reality.  Analyzing performance at an All-India level forgoes a more complete understanding for theoretical convenience -- India is a diverse nation with states that are at different levels of development, and a number of its characteristics may have prevented income convergence.  Prior to liberalization, Indian states had highly unequal levels of productivity, and the Indian Government's  centrally planned development strategy and public investment failed to eliminate these regional inequalities before market forces took over.  The private sector became the primary source of investment capital, and was drawn towards regions that already had productivity advantages, rather than underdeveloped areas that had potentially greater returns to capital.  Factor immobility and agglomeration economies caused growth to concentrate in already well-off states such as Maharashtra and Gujarat, leaving their less developed neighbors behind.  Instead of income convergence, reforms had uneven effects across India, and the divergence of incomes is widely found across states -- yet its causes are not well understood.  \par

This paper investigates the validity of the convergence hypothesis in India with a variety of measures.  The research joins a dense literature rife with theoretical uncertainty and empirical ambiguity, but its findings reinforce the growing consensus that India's state-wise growth does not fit the simple Solovian idea that poorer regions should grow faster than more developed ones.  The paper is primarily influenced by Sala-i-Martin's seminal (\citeyear{sala1996classical}) paper on convergence, and follows identification techniques from Ghosh (\citeyear{ghosh_economic_2008}), Baddeley et al., (\citeyear{baddeley_divergence_2006}), and Cherodian and Thirlwall (\citeyear{cherodian_regional_2015}).  An ordinary least squares (OLS) and generalized least squares (GLS) regression is performed using state fixed effects and time fixed effects, and evidence of conditional, but not unconditional, income convergence is found.  Differences in public investment, infrastructure, financial infrastructure, health, education, and sectoral composition are found to explain the variation in steady-state income levels across states. While a lack of data confines analysis to a segment of conditioning variables, the novelty of this study is its inclusion of detailed, recent data and its use of rigorous panel-data analysis.  The rest of the study is organized as follows: an overview of India's growth history is provided, followed by an analytical framework section with a run-through of stylized facts regarding growth determinants, a brief review of relevant literature, an econometrics identification strategy, a discussion of the results, and a conclusion.  While the focus of this study is income convergence across Indian states, the paper's motivating concern surrounds the waste of human potential that India will continue to experience if its problems of poverty and inequality remain unsolved.\par

\section{Indian Context}

The Indian economy is characterized by a paradoxical combination of crony capitalism and socialist impulses.\footnote{In 2016, Transparency International ranked it 79/176 on its Corruption Perception Index.  Bihar, the country's poorest state, earned the nickname 'Jungle Raj' for its famously corrupt government.}  Yet this struggle between socialism and primordial capitalist desire is a classical one,\footnote{\citep{unnikrishnan_2017}.} seen in countries like China and Vietnam today.  A long history of state-led industrialization and development followed by liberalization seem at odds, but the ensuing growth lifted millions out of poverty, from a poverty headcount ratio of 45\% in 1987 to 21.2\% in 2011.\footnote{World Bank Data, poverty headcount ratio at \$1.90 a day in 2011 PPP.}  From 1960 to 1980, India grew at a bleak ‘Hindu growth rate’ of about 3.5\% per annum, before increasing to 5\% in the eighties and upgrading further to 6\% in the nineties \citep{kotwal_economic_2011}.  Miraculously, the growth was not only fast, but relatively stable, rarely experiencing contractions.  Not everyone in India was included in this growth, however.  India is a diverse, federalist subcontinent with 29 states and seven union territories.  As such, effective analysis requires a more disaggregated view.  Regional inequality harms unity, so it is an important concern for a democracy whose power base lies in the wellbeing of people across all of its states.  Furthermore, India's poorest states are also its most populous, and if left untouched by growth, poverty and inequality in these regions could lead to serious economic, political and social difficulties in the future \citep{purfield_mind_2006}.  As this section will outline, the idiosyncrasies of India’s political economy and the socioeconomic inequalities inherent in its system since its genesis make it ill-suited for development that is both equitable and market-driven.\par
	Self-reliance is a concept that hearkens back to Mahatma Gandhi’s \textit{swaraj} during India’s independence movement.  It was the driving principle behind India’s import substitution model from its independence in 1947 until its liberalization in 1991.  The state was the preeminent source of investment throughout the period.  There was a belief that complete reliance on market forces would result in both the potentially fatal exposure of vulnerable rural populations to macroeconomic volatility, and under-investment in the sectors believed essential for sustainable growth and development.  What is more, many feared that integrating with the global economy would lead to increases in inequality.  \par
Strong incentives were given to capital intensive industries where India lacked comparative advantage \citep{kotwal_economic_2011}.  If it could be manufactured at home, many believed, it should be -- regardless of the cost.  Indian industry was stifled by government intervention in licensing, the use of foreign exchange, financial markets and prices.  A cornerstone of this system was the Industries (Development and Regulation) Act of 1951, or 'License Raj',\footnote{Named in lieu of the British Raj.} which strictly regulated entry and production in the manufacturing sector.  There was also substantial public ownership of heavy industries, and a labyrinthine trade regime with high nominal tariffs and non-tariff barriers as well as a complex import licensing system \citep{topalova_trade_2005}.  True to socialist style, India had a Planning Commission, including a system of Five Year Plans responsible for disbursing funds from the Central to State governments for a variety of targets.  Driven by equity and security concerns, directing Plan funds to underdeveloped regions was one of its central tenets, as articulated in Articles 280 (a) and (b) of the Indian Constitution \citep{ghosh_economic_2008,cherodian_regional_2015}.\par
    
\begin{figure}[!h]
\includegraphics[width=.65\textwidth, center]{state-and-union-territories-india-map}
\caption{Map of India}
\end{figure}

The federalist nature of India’s government deserves particular attention.  Prior to liberalization, the basic framework of the Central-State government relationship exhibited strong centripetal bias in powers.  Following the logic of a centrally planned economy, power lay overwhelmingly in the hands of the Central Government -- contradicting India’s federalist nature.  Broad-based taxes and the borrowing power of states both fall within the ambit of the Central Government \citep{kalirajan_fiscal_2012}.  Yet states are allocated lopsided financial responsibilities beyond their own revenue potential.  Their responsibility for health services and rural infrastructure means they are predominantly responsible for their poor.  Not only that, the abilities of each state to raise these funds are not equal.  The Finance Commission is responsible for filling these projected budget shortfalls, but Kalirajan and Otsuka (\citeyear{kalirajan_fiscal_2012}) argue that this practice fostered an irresponsible handling of funds by state governments.\par
India gradually became market-oriented in the mid-1980s with the liberalization of some trade and relaxation of select business regulations \citep{kalra_growth_2010}.  From the beginning, planners understood the potential gains from market-oriented reforms, but were hesitant given India’s behemoth and persistent issues of inequality and poverty, as well as the tendency of rapid structural adjustment to leave the most vulnerable behind.  In the second half of the 1980s, the strategy transitioned to export-led growth, accompanied by loosened import and industrial licensing regimes, and the replacement of quantitative trade restrictions with tariffs.  However, these were diluted measures, as the average tariff was still greater than 90 percent \citep{topalova_trade_2005}.  Growth accelerated following these measures, upgrading from the Hindu growth rate to a clipped pace of 5.82\% during 1980-1990 \citep{ghosh_regional_2012}.  This reform period was also marked by an increase in macroscopic imbalances as India's economy became susceptible to the ebbs and flows of the global economy.  Fiscal and balance of payments shortfalls accumulated until 1991, when the Indian Government resorted to the International Monetary Fund (IMF) for one of its contentious structural adjustment loans.\par
The loan was contingent on a comprehensive suite of reform measures, categorized as liberalization, deregulation, and privatization.  India sharply transformed from a mixed economy to a market-driven one, with fiscal policy reforms, financial sector reforms, liberalization of ‘the License Raj’, reforms in foreign trade and investment, infrastructure sector reforms, and agricultural sector reforms \citep{ghosh_regional_2012}.  The role of public investment and the Planning Commission were diluted.  According to neoclassical principle, these stabilization methods generate growth through the efficiency gains synonymous with the increased role of the free market.  Additionally, any benefits that are accrued should, according to theory, trickle down to the very poor.  Many will argue that the substantial growth following liberalization did successfully benefit both rich and poor states, and the careful measures taken to shield the rural areas from the shock treatment are emphasized \citep{jha_reducing_2002}.  However, a great deal of research finds that liberalization is responsible for exacerbating income disparities across states \citep{bhattacharya_regional_2004, kar_regional_2006,baddeley_divergence_2006, ghosh_economic_2008,kalra_growth_2010}.  \par
According to Pal and Ghosh (\citeyear{pal_inequality_2007}), tax revenue plummeted following liberalization, and the ensuing lack of public investment diminished aggregate demand.  This negatively affected private investments, creating infrastructure bottlenecks to future growth, and reducing the (already inadequate) distribution of public services.  With structural reform, resources are expected to be allocated by the market to states that have a comparative advantage.  Yet wide geographic, economic, and demographic disparities propagated from the failure of the Central Government to promote balanced development.  Specifically, the Planning Commission failed to offset significant fiscal disadvantages of states with low revenue capacity and high unit cost of providing public services before liberalization \citep{kalirajan_fiscal_2012}.  Private capital is drawn to profit-maximizing locations, oriented towards more developed regions, while allocation of public capital is based upon development and equity considerations \citep{lall_unu-wider_2015}.  But even the equity mindset of the Central Government’s investment decisions may be dubious.  A recent paper by Asher and Novosad (\citeyear{asher_politics_2013}) found that regional constituencies have significantly better performance under a ruling party candidate than under opposition candidates, while opposition constituencies actually experience weakened growth.  Cole (\citeyear{cole_fixing_2009}) also finds that agricultural credit from state banks follows an electoral cycle, targeting swing states and the rural farmer vote during regional elections.  Any aspect of investment (public or private) that requires government approval is a breeding ground for bribery.  Regardless, the dominance of private investment over public investment following liberalization and its tendency towards industrialized states played a role in increased regional disparities.  Purfield (\citeyear{purfield_mind_2006}) finds that capital went primarily to wealthier states.  After all, in his first budget speech as Finance Minister in 1991, Dr. Manmohan Singh correctly stated, 'But markets can only serve those who are part of the market system.  A vast number of people in our country live on the edges of a subsistence economy' \citep{singh1991budget}. \par
	The implementation of liberalization came largely out of a need for the IMF's aid, and did not necessarily originate from thorough analysis -- drawing many parallels to the recent 'shock therapy' policy of demonetization.\footnote{Indeed, one of the primary policymakers involved, Dr. Raja Chelliah, stated ‘We didn’t have time to sit down and think exactly what kind of a development model we needed … There was no systematic attempt to see two things; one, how have the benefits of reforms distributed, and two, ultimately what kind of society we want to have, what model of development should we have?’ \citep{topalova_trade_2005}. }   Reforms were non-discriminatory, points out Ahuluwalia (\citeyear{ahluwalia_state-level_2000}), but they had disparate effects on states due to their diverse fundamental characteristics.  However, whether liberalization was wrongly-conceived is not the focus of this paper -- rather, the behavior of inter-state income inequality following it is investigated instead.  According to neoclassical growth theory, outlined in the next section, the incomes of the poorest states should have accelerated and caught up with the richest states.  Instead, Purfield (\citeyear{purfield_mind_2006}) found that growth was most volatile in the poorest states.  Ahluwalia (\citeyear{ahluwalia_state-level_2000}) suggests that, although inter-state Gini coefficients increased in the post-reform period\footnote{From 0.152 in 1980 to 0.233 by 1999 \citep{ahluwalia_state-level_2000}.}, the idea that the rich states grew unambiguously richer while the poor grew poorer lacks is misleading -- the reality is more nuanced.  Punjab and Haryana were the richest states pre-reform, but post-reform their growth fell below the national average.  Even the behaviour of the BIMARU\footnote{A pun, named after the Hindi word for sick, \textit{bimar}.  Includes Bihar, Madhya Pradesh, Rajasthan and Uttar Pradesh.} states, first grouped because of their common demographics and poor performance, differed greatly following reforms.  Rajasthan and Madhya Pradesh performed reasonably well, while Bihar and Uttar Pradesh were among the poorest performers.  Maharashtra and Gujarat, which were ranked directly below Punjab and Haryana, experienced the highest growth in the post-reform period at 8\% per annum.  Overall, though, India's growth has been unbalanced, and is accompanied by rising inequality, a bias towards services, and excessive concentration of growth in a few large cities \citep{das_remoteness_2013}. \par

    
Balanced regional growth has always been a national policy objective in India, and it appears that economic reforms have done little to promote it.  These regionally disparate  effects are inconsistent with standard growth theory.  It is quite possible that all of these states have different ‘steady-state’ levels of per capita income according to their respective endowments -- the effect of unfelt until the advent of market reforms when the role of comparative advantage took center stage.  This is the motivation of the paper -- to find evidence of income convergence conditional on these diverse characteristics.  \par

\section{Analytical Framework}
\subsection*{Growth Theory}
Broadly, growth theory predicts a number of relative outcomes for regions’ incomes.  The existence of income convergence across countries was first proposed as a validity test for a number of these theories \citep{sala1996classical}.  Neoclassical growth theory predicts unconditional convergence to a single steady-state level of income per capita, where poorer regions grow unambiguously faster than richer regions.\footnote{Neoclassical growth theory can also assume non-decreasing returns to capital, depending on the paper, and predict conditional convergence in the Solow growth framework.  In the model papers, it is distinguished from endogenous growth theory regarding this assumption, and the technology parameter is considered a public good shared by all, resulting in unconditional convergence.}  Endogenous growth theory predicts conditional convergence, where regions converge to different steady-states based on differences in their production fundamentals.  Cumulative causation theory, on the other hand, predicts the existence of agglomeration economies, where economic growth propagates more growth, and failure brings more failure, causing income divergence.  Each possibility has palpable implications for inequality across states in India, ranging, respectively, from diminished or persistent inequality to inequality that increases indefinitely.\par
 More specifically, neoclassical growth theory predicts that the capital/labor ratio, or regional differences in income per capita, should converge to a common steady-state level \citep{solow_contribution_1956}.  Growth is based upon capital accumulation, and the growth rate of a region is positively related to the distance from its steady-state level \citep{sala1996classical}.  Otherwise known as exogenous growth theory, it rests on the assumptions of constant returns to scale, diminishing marginal returns to capital and, chiefly, factor mobility.  Factors move to where the returns are highest, equalizing wages and rents.  In other words, if the only difference between two regions is the initial capital level, then poorer regions should grow faster than wealthier ones; converging unconditionally \citep{sala1996classical}.  If a national economy is well integrated, regions should display strong tendencies towards income convergence as those with initially lower incomes catch up.  Hence, regional disparities are seen as growing pains, eventually resolved by equalizing market forces which act to minimize the distance between leading and lagging regions.  \par
	 Romer (\citeyear{romer_increasing_1986}) and Lucas (\citeyear{lucas_mechanics_1988}) developed ‘new’ endogenous growth theory to expand upon neoclassical growth theory.  Both economists call in to question the validity of the diminishing returns to capital assumption, identifying forces that increase the productivity of capital outside of Solow’s ambiguous technical progress parameter.  Development of human capital and technology prevents factor productivity from decreasing as countries grow wealthier, and technology may exhibit increasing returns to scale.  Endogenous growth theory also relaxes the assumption that countries only differ in their initial levels of capital, and accounts for differences in technology, population growth, and production functions.  Separate states might converge to different steady-state levels of income per capita, but the disparities between these steady-states could persist, narrow or even widen with a change in the distribution of their structural characteristics.  To find evidence of conditional convergence, a variety of factors must be controlled to ‘hold constant the steady-state of each economy’, after which there should be evidence of a negative partial correlation between income growth and initial income level \citep{sala1996classical}.  Poorer regions will grow faster than wealthier ones conditional on sharing the same steady-state.  The implication of this ‘conditional convergence’ is that a country’s production fundamentals drive income per capita -- not initial levels of income or capital.  However, as stressed by Sala-i-Martin (\citeyear{sala1996classical}), this does necessarily mean that poorer regions are always growing faster than wealthier ones, or that overall income inequality is expected to shrink.  This would be the case if there were evidence of unconditional convergence, or convergence of income levels ($\sigma$ convergence).  Conditional convergence implies that economies converge on a stratified distribution of steady-state levels of income brought about by their own production fundamentals, and growth slows as the economy draws closer to this level.\par
These orthodox theories are confronted by Gunnar Myrdal in his book Economic Theory and Underdeveloped Regions (\citeyear{myrdal_economic_1957}).  His theory of cumulative causation is modeled in part upon ‘Verdoorn’s Law,’ where faster growth in output increases productivity due to increasing returns, making growth both circular and cumulative.  He acknowledges the ‘virtuous circle’ in developed countries and the ‘vicious circle’ in developing countries.  Higher capital/labour ratios will draw more capital due to a variety of productivity advantages and spillovers, and poorer regions may never catch up to wealthier regions.  Myrdal claims that orthodox theory conflates the role of factor migration in equalizing these regional disparities entirely.  \par
Instead of equalizing factor prices, economies of scale in industrial centers will counteract the movement of production to less developed areas, and will actually lead to an agglomeration of skilled labor, capital, industries and demand, thereby fortifying regional disparities.  Increasing returns to scale from intangible aspects of urban infrastructure such as easy communication, transport, knowledge spillovers, access to markets and resources, and entrepreneurial spirit similarly preserve these growth patterns.  These types of factors provide a significant competitive advantage, are difficult to measure, but are above all highly immobile.  \par
Regions untouched by ‘expansionary momentum’ cannot afford to maintain good infrastructure and public utilities, causing their competitive disadvantages to mushroom.  Expanding on the case of human capital, poorer regions could struggle to afford healthcare and education initially, stunting their labor productivity and competitiveness at the outset.  Banerjee and Duflo (\citeyear{banerjee_growth_2005}) also indicate that those with potentially high returns to capital may not have access to credit.  Indian bank managers show abnormal risk aversion in lending, even to medium-sized firms \citep{banerjee_what_2004}.  Instead of converging, income per capita across regions could actually diverge indefinitely given unequal initial (human) capital endowments and constraints.  These non-economic factors are core vehicles of this cumulative causation, the exclusion of which ‘represents one of the principle shortcomings of economic theory’ \citep{myrdal_economic_1957}. \par 
Non-orthodox theory of Myrdal’s like dismisses the predictions of convergence as simplified and static, ignoring the realities of underdeveloped regions beyond their observable productive potential.  These realities are distinctly relevant to a case such as India, where capital constraints and a lack of infrastructure represent just some of the barriers to growth overlooked by neoclassical economics.  The idea of diminishing returns is particularly inapplicable to India, where for decades public investment was predominant and guided by non-market factors \citep{ghosh_economic_1998}.  As Cherodian and Thirlwall (\citeyear{cherodian_regional_2015}) point out, ‘It is an interesting question why, in the teaching of regional growth and regional disparities, the neoclassical prediction of convergence has always been the initial presumption, rather than the non-orthodox prediction of divergence, but that is a question for historians of thought to answer.’
 

\subsection*{Growth Determinants}

	The primary step in conditional convergence analysis is selecting the variables to proxy for different steady states.  In Solow's neoclassical theory, savings allows for investment and capital accumulation.  This section expands upon this with stylized facts about growth in the context of India.  While in neoclassical growth theory, convergence is the outcome of market forces left untouched by government intervention, public spending is found to be a significant engine of growth for some of India’s poorest states.  The Central Government played a central role in state-led development prior to liberalization, and the Planning Commission was an important institution in coordinating the transfer of resources from the Central to State governments.  Many states rely on the Central Government for their infrastructure.  Funds distributed by the Planning Commission for development, namely state-wise Per Capita Plan Outlay, would have benefited the analysis by examining the role of the government in explicitly targeting regional disparities, but the data needed is not publicly available.  Nevertheless, Ghosh et al. (\citeyear{ghosh_economic_1998}) find a negative relationship between the disbursement of plan funds and state income level, confirming that planners have targeted poorer states.  \par
    Physical capital accumulation is also essential to growth and its absence is a key constraint on productivity and attracting future investment.  The role of infrastructural investment is especially relevant in the Indian context, particularly electric power, road and rail, telecommunications and irrigation \citep{baddeley_divergence_2006}.  In 2005, 38\% of Indian firms claimed that access to quality transport or power infrastructure was a major obstacle to growth, and many firms resort to private provision of these goods \citep{asher_politics_2013}.  Ghosh et al. (\citeyear{ghosh_economic_1998}) argue that regional imbalances in physical infrastructure have been responsible for increasing income disparity across states.  The true supply of infrastructure is also important to proxy.  Larger public infrastructure in one region is not guaranteed to lure private capital given potential inefficiencies.  Indeed, it is entirely possible that a share of public projects in India are simply roads to nowhere.  In one case, NGO activists were sent to follow up on a local government’s recently built dams.  The local officials had built only one dam, but billed for it four times, and drove the activists to the same dam using four different paths in an attempt to fool them \citep{pritchett_review_2009}.  \par
Following liberalization and the diminished role of the government in equalizing regional growth, the private sector became the primary source of new investment, making financial infrastructure a vital engine of growth.  Private capital accumulation is not dictated by the equity and security considerations of public planers.  States now compete for investment, potentially exacerbating inter-state inequality.  Some states have done well in this regard, including Tamil Nadu, Karnataka, Andhra Pradesh, Gujarat and Maharashtra, attracting car manufacturers such as Ford and software companies such as Microsoft \citep{bajpai_trends_1996}.\footnote{According to Purfield (\citeyear{purfield_mind_2006}) Maharashtra, Gujarat, and Tamil Nadu alone produce about 40\% of industrial and service sector output.}  Not all states have shared this success, and spatial inequality of industry is expected to be a primary determinant of inter-state wealth disparities.  Unfortunately, there is a lack of data on private investment, so this paper will examine state-wise financial infrastructure instead.\par
The positive relationship between growth and human capital is robust in a number of empirical studies on conditional convergence \citep{mankiw_contribution_1992}.  Inter-state variation in education initiatives is expected to be quite high, as it lies in the foray of State Governments.  Gender inequality and caste explain many of the inequalities in public school provision across states \citep{baddeley_divergence_2006}.  Kerala’s distinct improvements in education and health are frequently cited as part of the secret to its economic success.  Human capital is expected to play an essential role post-liberalization due to the rise of the services sector and the increasing importance of skill across all sectors.  Both health and educational aspects of human capital are expected to have a positive influence on growth.  \par
India’s inter-state inequality is the essence of this paper, yet a lack of state-wise Gini coefficients and measures for gender inequality preclude analysis.  Inequality is commonly considered a drag on growth, but some schools of thought believe it drives growth through incentive and savings channels \citep{stewart_income_2012}.  Vanneman and Dubey (\citeyear{vanneman2010horizontal}) find that within India, higher-income states have almost the same average levels of inequality as lower-income states.  They also find that state differences in income levels account for only a minor proportion of national income inequality: most inequality exists within states.  Gender inequality would also be a useful proxy, given the subordinate role of women in India’s labor force, but there is a lack of state-level data of this nature to choose from.  Most papers use female school enrollment, but surprisingly the literature has found that gender inequality does not explain much of the income differentials in India \citep{purfield_mind_2006,baddeley_divergence_2006}.  This could be because many educated females do not join the labor force, a reflection of some of the top performing states’ (Haryana and Punjab) well-known preference for sons, or because gender inequality has a collinear relation with some of the other variables that are expected to detract from growth.  \par
 Sectoral composition of output will be included in the model because economic activities such as agriculture are diminishing returns activities that slow labour productivity growth unless offset by technical progress.  Poorer states are expected to have higher shares of agriculture in their production structure.  An assumption in the convergence hypothesis is that most workers in poor economies are initially in low productivity sectors such as agriculture, and then transition to higher productivity sectors following economic growth- a process known as structural transformation.  Nonetheless, India’s economy is characterized with an extremely slow process of labor transfer between sectors \citep{jayanthakumaran_economic_2010,kotwal_economic_2011,topalova_trade_2005}.  Two exceptions to this rule are Haryana and Punjab, which benefited immensely from the technological progress from the Green Revolution and their resulting increased agricultural productivity \citep{baddeley_divergence_2006}.  Both remain quite agricultural, yet they are two of the fastest growing Indian states.\footnote{However, the bulk of the agricultural workforce consists of migrants, whose factor incomes are not reflected in their state of residence, so results may not reflect this aspect \citep{purfield_mind_2006}. }   \par
	The effect of population growth on economic performance has been debated since Malthus, and plays a central role in growth.  The Solow steady-state model predicts that an economy must save enough each year to maintain the steady-state level of the capital/labor ratio as population grows.  A higher birth rate implies more savings needed simply to maintain the steady-state, and the resulting high dependency ratio consumes a significant proportion of a population’s savings.  Thus, a fall from India’s historically high fertility rates could translate to a demographic dividend whereby a larger working-age population and falling dependency ratio results in higher savings rates per capita.\par
	While there is insufficient data to proxy factor mobility, its role is central to convergence theory.\footnote{Inter-state migration rates are not freely available for the entire panel.}  Despite significant urbanization trends within states, large migrations of labor across states has been conspicuously absent in India, possibly because of ethnic and linguistic gaps as well as poverty constraints \citep{kotwal_economic_2011}.\footnote{India is not only ethnically diverse, it is extremely linguistically diverse: 122 languages from 4 different language families are spoken by more than 10,000 people.  Languages from the North and South are often mutually unintelligible.}  According to Purfield (\citeyear{purfield_mind_2006}, only 6\% of migration in rural areas and 20\% in urban areas occurred across state borders.  Cashin and Sahay (\citeyear{cashin_internal_1996}) find that migration was also unresponsive to cross-state income differentials.  This could relate to the evidence of larger inequalities within states, where factor allocation could be confined to.  Desmet (\citeyear{desmet_spatial_2012}) argues that frictions, policies and a general lack of infrastructure in medium-density cities is preventing the spread of growth outside of high-density cities.  The lack of geographical mobility is accompanied by a lack of inter-sectoral mobility, as mentioned earlier.  Rigid labor markets are reinforced with complex regulations and contribute to the lack of factor reallocation in India.  An infamous example is the Industrial Disputes Act of 1947, which requires companies with more than 100 workers to seek written permission from a state government to close a plant or retrench workers \citep{fallon1991impact}. As such, productive companies may be constrained to smaller sizes to avoid red tape.  

\section{Convergence and Divergence in India}

Previous studies differ in their identification strategies and the data used, but broadly there is evidence found for conditional convergence across states in India.  A brief visual of the lack of evidence on unconditional $\beta$ and $\sigma$ convergence can be found below.  Most papers investigating conditional convergence find evidence in support of it.\par

\begin{figure}[h]
\includegraphics[width=\textwidth, center]{lit_review}
\caption{\citep{cherodian_regional_2015}}
\end{figure}

There are a number of approaches to testing for convergence, the first involving the use of cross-sectional data.  Cashin and Sahay (\citeyear{cashin_internal_1996}) use a cross-sectional regression while controlling for sectoral share of output to find evidence of unconditional convergence of states owing to the transfer of grants from the Central Government.  However, their coefficients are statistically insignificant.  Bajpai and Sachs (\citeyear{bajpai_trends_1996}) find divergence in both pre- and post-reform periods while controlling for initial agricultural share of output.  Ahluwalia (\citeyear{ahluwalia_state-level_2000}) uses cross-sectional data to investigate the main engines of income divergence, and finds that variations in private investment are positively and significantly correlated with growth disparities, while public and plan expenditure seem to have little bearing.  Baddeley et al. (\citeyear{baddeley_divergence_2006}) perform a cross-sectional analysis and find that, in absolute terms, the poorest states grew at slower rates than wealthier states, showing evidence of dispersion of income across states from 1970-1997.  They take advantage of both OLS and GLS techniques after finding evidence of heteroscedasticity, and control for a number of possible factors, including economic structure and physical and human capital formation.  They do observe conditional convergence, and attribute the value of state level investment and the promotion of agricultural productivity as important factors.  Their research also indicates that the liberalization measures of 1991 significantly intensified growth differentials among states.  Cherodian and Thirlwall (\citeyear{cherodian_regional_2015}) examine the period from 1999-2011 and find weak evidence of conditional convergence controlling for population growth, credit growth, male literacy, the share of agriculture in State Domestic Product (SDP), and state expenditure as a share of state GDP.  However, these regressions rely on the assumption that there are no state-level fixed effects, relying on time-invariant measures at fixed points in time instead.  They suffer from two principal sources of bias: unobserved state-specific heterogeneity and endogenous explanatory variables.  They assume homogeneous production functions across states, and ignore the heterogeneity of initial endowments, which carry considerable implications regarding estimation of unbiased convergence rates \citep{weeks2003provincial}. \par

Many papers \citep{nagaraj_long-run_2000,aiyar_growth_2001,purfield_mind_2006,kalirajan_economic_2010,cherodian_regional_2015} consider state fixed effects using panel data.  Aiyar (\citeyear{aiyar_growth_2001}) makes use of state annual income five years apart in order to control for short run fluctuations, and finds evidence of conditional convergence while controlling for literacy rate and real private sector credit per capita.  He finds that high literacy rates appear to support growth, contrary to some studies find them to be statistically insignificant \citep{purfield_mind_2006,kalirajan_economic_2010}.  Purfield (\citeyear{purfield_mind_2006}) finds evidence of unconditional convergence and conditional convergence as well as a strong positive correlation between growth rate and the share of the services sector, credit growth, and improvements in infrastructure.  She controls for time and fixed effects, and uses five year averages for her data.  Nagaraj et al. (\citeyear{nagaraj_long-run_2000}) find evidence of conditional convergence, and attribute it to differences in physical, social, and economic infrastructure endowments.  They control for the share of the agricultural sector, relative price shocks, and infrastructure variables, and use both time and fixed effects.  Kalra and Sodsriwiboon (\citeyear{kalirajan_economic_2010}) discover evidence of divergence, but find that the number of telephone lines is the only significant variable.  Ghosh (\citeyear{ghosh_economic_2008}) finds that the reason for increased income divergence has been state variations in production structures, human capital and infrastructure.  Abler and Das (\citeyear{abler_determinants_1998}) actually observe inconsistencies in income convergence across India, finding convergence in eastern India, divergence in north-eastern India, and a lack of any tendency towards convergence in central and south India, or the All-India level between 1961-1990. Sachs et al. (\citeyear{sachs2002understanding}) does not find evidence of conditional convergence, finding instead that 82\% of variation in regional growth is explained by urbanization rates.  \par

Literature examining regional disparities in India is not limited to convergence models.  Lall and Chakravorty (\citeyear{lall_unu-wider_2015}) find increasing spatial inequality in industrialization, and show that new private sector investments in India are biased towards existing industrial and coastal districts.  State industrial investments, which have diminished following structural reforms, are far less biased towards such districts.  Kar and Sakthivel (\citeyear{kar_regional_2006}) find that the growth of the sectoral share of services and industry was largely responsible for the increased income inequality during the period, while income growth in the agricultural sector offset some of the divergence. Baddeley et al. (\citeyear{baddeley_divergence_2006}) find that poverty declined but became more spatially concentrated following liberalization.  Many papers identify an acceleration in regional inequality following structural reform in the 1990s \citep{jha_reducing_2002,chakravorty_capital_2003,kar_regional_2006}.\par

\section{Econometric Model}

	This study will take advantage of panel data from 1991-2011, benefiting from the use of state fixed effects and time fixed effects.  The identification strategy will follow the ‘Barro regression method’ suggested by Barro and Sala-i-Martin (\citeyear{barro_convergence_1992}), where growth is expressed as a function of initial income and determinants of the steady state.  This methodology is used widely in the literature \citep{purfield_mind_2006,baddeley_divergence_2006,kalra_growth_2010,ghosh_regional_2012,cherodian_regional_2015}.  This enables not only the testing of convergence but an investigation of factors that affect economic growth as well.  The empirical hypothesis of unconditional income convergence across Indian states will be tested, and an investigation of any tendency towards conditional convergence will be conducted.  This research will build on previous research by including a comprehensive selection of control variables that focus on distinct aspects of growth theory.\par
Convergence derives from the neoclassical property of diminishing returns to scale, and is broadly measured in two ways: $\beta$ and $\sigma$ convergence \citep{sala1996classical}.  $\beta$ convergence examines the mobility of states within a country’s income distribution, while $\sigma$ convergence concerns the disparity of state incomes over time.  The existence of $\beta$ convergence generates $\sigma$ convergence; if poorer states are growing faster than richer states, then the levels of SDP should equalize over time.  Yet unconditional $\beta$ convergence is not sufficient for $\sigma$ convergence to occur because of the potential for random shocks, and conditional $\beta$ convergence is insufficient because of the possibility that steady-state levels of SDP may diverge through an increased dispersion of conditioning variables. \par   
$\sigma$ convergence is traditionally measured by examining the cross-sectional coefficient of variation (CV), or relative standard deviation, of income over time to test whether disparities increase or decrease.  This paper will graph the results over time a la Ghosh (\citeyear{ghosh_economic_2008}). 
$$ \sigma_{t+T} < \sigma_t$$
Where $\sigma_t$ is the standard deviation of $\log(y_{i,t})$ across i states, and $\sigma_{t+T}$ is the standard deviation T periods later.  The coefficient of variation should decrease over time if states are converging.  This measure of dispersion does not capture income dynamics.\par
  Within $\beta$ convergence, there exists either conditional or unconditional convergence.  Both imply mean reversion, where states grow slower as they become wealthier, but conditional convergence predicts that regions will converge on different steady states depending on fundamental regional differences.  The unconditional convergence approach involves a regression of each state’s income per capita growth on initial income per capita.  Our model estimates the following equation using the ordinary least squares (OLS) method.  
$$ \gamma_{i,t,t+T} = \alpha + \beta\log(y_{i,t}) + \epsilon_{i,t}$$

Where $\gamma_{i,t,t+T} \equiv \log(y_{i,t+T}/y_{i,t})/T$, or the ith state’s annual growth rate of income per capita in the time period, and $\log(y_{i,t})$ is ith state’s income per capita at the beginning of the sample period.  For unconditional convergence, the coefficient on the logged initial income per capita ($\beta$) is expected to be significantly negative, capturing the inverse relationship between growth rate and initial income.  Cross-sectional average income growth rates and initial income levels over the sample period are used, similar to Ghosh (\citeyear{ghosh_economic_2008}).  However, as outlined in the previous section, there are many growth determinants that are likely omitted.\par   
Conditional convergence hypothesizes that regions may not share a single steady-state income level, and, conditional on structural differences, have different steady-states that run parallel to each other.  The conditional convergence approach simply adds a vector of controls for heterogeneity across states, represented by $\Psi X_{i,t}$.  Using panel data estimation techniques, the OLS and GLS methods will be used for the entire time period.  Instead of using averages from the period, this strategy will use annual growth rate $\gamma_{i,t,t+T}$ along with $\log(y_{i,t-1})$. 

$$ \gamma_{i,t,t+T} = \alpha + \beta\log(y_{i,t-1}) + \Psi X_{i,t-1} + \epsilon_{i,t,t+T}$$

Conditional convergence exists if $\beta$ is significantly negative.  The panel data will be estimated on an annual basis.  In this paper, the choice of conditioning variables depends on economic theory involving their association with increasing returns, outlined in the previous section, and the availability of data, most of all.  This study will include population growth as a control for demographic trends, social expenditure to control for government spending, gross fixed capital formation to control for infrastructure investment, personal loans by scheduled commercial banks\footnote{Nearly 75\% of all financial assets in India are accounted for by SCBs, so this is an appropriate proxy for financial infrastructure \citep{cherodian_regional_2015}.} to control for financial infrastructure, the literacy rate to control for educational human capital, per capita electricity consumption to control for infrastructure stock, infant mortality rate to control for non-educational human capital, and the percentage share of agriculture in SDP to control for economic structure.\footnote{Social expenditure, gross fixed capital formation, and personal loans are all logged and expressed as a percentage of SDP.}  This is not an exhaustive list, more variables were available -- and are likely important, but these were prioritized in the interest of a parsimonious model.  The expected signs on infant mortality rate and percentage share of agriculture are negative, while the expected signs on social expenditure, gross fixed capital formation, personal loans by SCBs, literacy, and electricity consumption are positive.  The sign on population growth cannot be determined \textit{a priori}, due to the conflicting theories about its role in growth.\par 
As outlined in Baltagi (\citeyear{baltagi_econometric_2008}), panel data gives more information, more variability, less multicollinearity, more degrees of freedom and more efficiency.  However, panel data is prone to cross-sectional dependence, where states’ fortunes may be tied to one another, and the effect of measurement error is amplified.  To prevent cross-sectional dependence, variables are instrumented by their lagged values and are assumed to be exogenous.  An additional source of bias and inconsistency in panel data is non-stationarity from the inclusion of a temporal dimension.  All variables were tested for a unit root using the Augmented Dickey-Fuller test, and residuals were tested for normality.  While some of the variables were found to follow AR(1) processes, the residuals of the regression were found to be co-integrated, so no differencing was necessary.  \par
The principal advantage of panel data, however, is the ability to control for unobservable factors, or variables that are state-invariant or time-invariant.  That is, it accounts for individual heterogeneity.  Time-series methods for panel data are especially appropriate for Indian states, as the short data span is actually compensated for by cross-sectional variation \citep{kalra_growth_2010}.  Studies that rely solely on cross-sectional data without controlling for this risk omitted variable bias from factors that are correlated with the explanatory variable and included in the error.  In the case of convergence studies, it is quite likely that growth is correlated with factors in the error term that are omitted from the model.  In response to these statistical issues, a common technique with panel data is fixed effects estimation.  In a fixed effects model, the error is treated as a number of unknown coefficients, jointly estimated with $\beta$.  These different initial fundamentals are captured with dummy variables to absorb individual fixed effects particular to each Indian state, allowing them to have distinct but parallel steady states.\footnote{“The key insight is that if the unobserved variable does not change over time, then any changes in the dependent variable must be due to influences other than these fixed characteristics.” \citep{stock_introduction_2003}}  By parsing out these unobservable effects, fixed effects estimation hopes to study the net effect of regressors on the dependent variable.  An important assumption of the fixed effects model is that state fixed effects are uncorrelated with each other, and are unique to each state.\par 
A Fisher test was first conducted to confirm the appropriateness of the fixed effects method and reject the null of a common intercept for all states.  Robustness checks for functional form,\footnote{The Ramsey RESET test was used, which examines whether non-linear combos of the fitted values can better explain the dependent variable.  The null hypothesis for correct functional form was rejected, but this is not an issue since it is quite likely that convergence models do not adequately capture inter-regional growth patterns \citep{baddeley_divergence_2006}.} heteroscedasticity\footnote{Breusch-Pagan test for heteroscedasticity was used.} and serial correlation\footnote{Breusch-Godfrey test for serial correlation was used.} were performed in the spirit of Baddeley et al., (\citeyear{baddeley_divergence_2006}) and Cherodian and Thirlwall (\citeyear{cherodian_regional_2015}).  Evidence of both heteroscedasticity and serial correlation were found, prompting the use of the generalized least squares (GLS) method\footnote{The GLS method produces more reliable and efficient results than OLS methods in the presence of heteroscedasticity by relaxing the assumption that residuals are normally distributed.} in addition to OLS, and robust standard errors clustered at the group level.\footnote{Without clustering, standard errors would be biased downwards, so the likelihood of a coefficient being statistically significant increases.}  A Hausman test was conducted to test the suitability of the random effects model.  The null hypothesis of random effects was rejected, so a fixed effects model is used.   Specifications with fixed effects and fixed effects including year-wise dummy variables to account for time-specific effects are included.\footnote{A joint test was performed to test for time-fixed effects, and the F score was significant.} \par

\subsection*{Data \& Variables}

Detailed information on data sources and transformations can be found in the appendix.  The primary source of data was the Reserve Bank of India’s Handbook of Statistics on Indian States, which provides detailed state-level data from the early 1990s onward.\footnote{Regretfully, there is a lack of data other than SDP before 1991, which prevents a detailed analysis of the effects of liberalization reforms on conditional convergence.}  India’s 15 largest states (Andhra Pradesh, Assam, Bihar, Gujarat, Haryana, Karnataka, Kerala, Madhya Pradesh, Maharashtra, Odisha, Punjab, Rajasthan, Tamil Nadu, Uttar Pradesh and West Bengal) are used for analysis as they make up more than 80\% of India’s GDP.\footnote{From own calculations in year 2011-12.} \par  As for the data itself, a quote taken from an economist who has studied India’s economy for years provides an apt summary of the problematic nature of Indian data, as well as the appropriate attitude when confronted by it:\par

\begin{displayquote}[!t]
‘Ideally, the SDP data series for individual states would be fully consistent with the national accounts estimates of GDP but this […] is not possible at present.  Information on the SDP […] is collected by the CSO […].  In this process the CSO takes note of differences in methods of estimating the SDP in different states, but it does not refine the SDP series to make them consistent with each other and with the national accounts. […] Following established academic tradition, we […] acknowledge the problem, but proceed undeterred’ \citep{ahluwalia_state-level_2000}
\end{displayquote}\par
Similar to many studies of developing countries, inferior data quality seriously handicaps this analysis from the start.  Despite collecting economic data for decades, India’s data is especially deficient.  Many statistics are collected at a state level using different methodologies, and some are clearly incorrect.  In a famous instance, Deaton and Dreze (\citeyear{dreze_poverty_2002}) discuss the ramifications of the change in questionnaire design of the 55th Round of the poverty-based National Sample Survey (1999-00), rendering it incomparable to the previous round.  A change in the recall period for durable goods from 30 days to 365 days meant respondents reported more purchases, driving the all-India headcount ratio down from 36\% in 1993-94 to 26\% in 1999-2000.  This was an especially important period of observation due to the liberalization reforms in 1991, and despite the well-known shortcomings of the 55th Round, the Indian Government used the figures as official estimates.  It is quite likely that there are numerous discrepancies along the same vein.  Missing values, in particular population, had to be interpolated in order to create interpretable per capita variables.\footnote{This was done in a number of studies, including Kotwal (\citeyear{kotwal_economic_2011}).  In RStudio, the approx() function used introduces a tiny amount of numerical imprecision.}  As Ghosh et al. (\citeyear{ghosh_economic_1998}, p. 1624) point out, ‘it is naïve not to incorporate inter-state price differentials in deriving state-wise real income because it can have substantial impact on the results of convergence.’  However, data constraints prevent the inclusion of a state-wise inflation regressor as well as statistics for private investment, yet a large part of the literature carries on with analysis despite this data deficiency. \par

\section{Results}
The results here are only representative of the sample period, and require a degree of scrutiny thanks to the potential sampling errors mentioned earlier.  \par

\begin{figure}[!htbp]
\includegraphics[width=\textwidth, center]{income_ggplot}
\caption{Income per Capita Divergence}
\end{figure}
\par

\begin{figure}[!htbp]
\includegraphics[width=\textwidth, center]{better_bump}
\caption{State Ranking of Income per Capita Across Time}
\end{figure}
\par


Figure 3 provides a cursory observation of the disparity in per capita income across Indian states, and its conspicuous widening in the past 20 years since liberalization in 1991.  Though inconclusive, evidence of unconditional convergence seems unlikely, supporting the findings of previous studies \citep{bajpai_trends_1996,ghosh_economic_1998,nagaraj_long-run_2000}.  Figure 4 demonstrates the lack of mobility between three broad groups of states, the rich (Gujarat, Maharashtra, Haryana and Punjab), the middle (Tamil Nadu, Kerala, Karnataka, Andhra Pradesh, Rajasthan, West Bengal, and Madhya Pradesh\footnote{Although Madhya Pradesh has experienced quite low income growth and has fallen through the rankings.}), and the poor (Odisha, Assam, Uttar Pradesh, and Bihar, which has witnessed an especially abysmal performance since the start of the sample period). A handful of studies investigate the existence of club convergence, or convergence of a group of states to a few steady-state levels \citep{baddeley_divergence_2006,ghosh_economic_2008,cherodian_regional_2015}.  However, this is outside the scope of this study.  The lack of data prior to 1991 is lamentable, as the possibility of a structural break in the data would also be interesting to study, but many papers find evidence that liberalization was indeed a conduit for increasing income inequality \citep{rao_convergence_1999,baddeley_divergence_2006,ghosh_economic_2008,kalra_growth_2010}.  While all states grew, they grew at extremely disparate rates.  In 1991, the richest state, Punjab, had about 2.7 times more income than the poorest state, Odisha.\footnote{That gap actually narrowed in 2011 to about 2.1 times, with Punjab experiencing a moderate average growth of 4.78\% in the first half of the sample and slowing to 3.56\% in the second half, and Odisha growing faster at 3.56\% then 5.68\%.}  The second poorest state in 1991, Bihar, grew a measly 2.88\% in the 90s and even scaled up to over 5\% in the second half, but still went from having 2.2 times less income than Punjab in 1991 to having 3.2 times less income in 2011 -- and Punjab is not even the wealthiest state anymore.  The gap between Bihar and the richest state in 2011, Gujarat, was actually fourfold.  States like Rajasthan and Kerala experienced enormous income gains in the period from a relatively poor beginning, but this is not a rule, as the fastest growing states were overwhelmingly the richest.  Overall, the highest growth rates were achieved by states evenly distributed throughout the initial income distribution.  \par

\begin{figure}[!htb]
\includegraphics[width=.7\textwidth, scale = 0.5, center]{sigma_convergence}
\caption{$\sigma$ Divergence}
\end{figure}
\par
Figure 5 presents the evolution of the coefficient of variation of log income per capita levels across Indian states from 1991-2011 in a test for $\sigma$ convergence.  Unambiguous $\sigma$ divergence is displayed as the cross-state dispersion of income levels widened during the period, despite fluctuations in some years.  Previous research \citep{baddeley_divergence_2006,ghosh_economic_2008,ghosh_regional_2012} confirms that this process has accelerated since the 1991 reforms, and that the process has been occurring since independence.
 Figure 6 presents a \textit{prima facie} examination of the relationship between the growth rate of income and initial income, with the average growth rate for each state from 1991-2011 graphed against the initial 1991 income of each state.  Despite weak evidence of unconditional convergence displayed by the trend line, the variance of the points makes this relationship statistically ambiguous.  \par


\begin{figure}[!h]
\includegraphics[width=.8\textwidth, center]{unconditional_chart}
\caption{Unconditional $\beta$ Convergence}
\end{figure}
\par



\begin{table}[!htb] \centering 
  \caption{Unconditional Convergence} 
  \label{} 
\begin{tabular}{@{\extracolsep{5pt}}lc} 
\\[-1.8ex]\hline 
\hline \\[-1.8ex] 
 & \multicolumn{1}{c}{\textit{Average Growth of per capita SDP}} \\ 
\cline{2-2} 
\hline \\[-1.8ex] 
 log(Initial SDP per capita) & $-$0.002 \\ 
  & (0.007) \\ 
  & \\ 
 Constant & 0.129 \\ 
  & (0.103) \\ 
  & \\ 
\hline \\[-1.8ex] 
Observations & 15 \\ 
R$^{2}$ & 0.005 \\  
\hline 
\hline \\[-1.8ex] 
\textit{}  & \multicolumn{1}{r}{$^{*}$p$<$0.1; $^{**}$p$<$0.05; $^{***}$p$<$0.01} \\ 
\end{tabular} 
\end{table}
\par

Table 1 displays the parameter estimates of our test for unconditional convergence in an attempt to explain observed divergence of per capita income.  Poorer states are hypothesized to grow faster than wealthier states, but an OLS regression of the average annual growth of income on initial income shows the lack of correlation between the two, as $\beta$ is not significantly different from zero.  It is highly unlikely that states are converging to a common steady-state income level.  The R$^{2}$ is also extremely low, however these findings suffer from a very small sample of the 15 states in the study.  The empirical findings are not surprising, given the inadequacy of unconditional convergence models in capturing the evolution of SDP, but differ from many studies that find evidence of unconditional divergence \citep{ghosh_economic_2008,nagaraj_long-run_2000,aiyar_growth_2001,ghosh_regional_2012}.  
The panels in the next two pages provide a preliminary illustration of the relationship between income and factors thought to facilitate spatially equitable development.\footnote{All income levels in rupees (\rupee).}  Notably, social expenditure has a negative correlation with income level, which may reflect a poorer state's relatively disproportionate spending on social infrastructure and benefits.\par


\begin{figure}[!hp]
    \centering
    \captionsetup{justification=centering}
    \begin{minipage}{0.45\textwidth}
        \centering
        \includegraphics[width=0.9\textwidth]{population_growth} % first figure itself
        \caption{Population vs Income Growth}
    \end{minipage}\hfill
    \begin{minipage}{0.45\textwidth}
        \centering
        \includegraphics[width=0.9\textwidth]{social_growth} % first figure itself
        \caption{Social Spending vs Income Growth}
    \end{minipage}\hfill
    \begin{minipage}{0.45\textwidth}
        \centering
        \includegraphics[width=0.9\textwidth]{fixed_capital_growth} % first figure itself
        \caption{Infrastructure Investment vs  Income Growth}
    \end{minipage}\hfill
    \begin{minipage}{0.45\textwidth}
        \centering
        \includegraphics[width=0.9\textwidth]{loan_growth} % first figure itself
        \caption{Personal Loans vs Income Growth}
    \end{minipage}\hfill
    \begin{minipage}{0.45\textwidth}
        \centering
        \includegraphics[width=0.9\textwidth]{literacy_growth} % first figure itself
        \caption{Literacy vs Income Growth}
    \end{minipage}\hfill
    \begin{minipage}{0.45\textwidth}
        \centering
        \includegraphics[width=0.9\textwidth]{elec_growth} % first figure itself
        \caption{Electricity vs Income Growth}
    \end{minipage}\hfill
    \begin{minipage}{0.45\textwidth}
        \centering
        \includegraphics[width=0.9\textwidth]{imr_growth} % first figure itself
        \caption{Infant Mortality vs Income Growth}
    \end{minipage}\hfill
    \begin{minipage}{0.45\textwidth}
        \centering
        \includegraphics[width=0.9\textwidth]{agshare_growth} % second figure itself
        \caption{Share of Ag. vs Income Growth}
    \end{minipage}
\end{figure}

\clearpage

\begin{figure}[!hp]
    \centering
    \captionsetup{justification=centering}
    \begin{minipage}{0.45\textwidth}
        \centering
        \includegraphics[width=0.9\textwidth]{population_levels} % first figure itself
        \caption{Population vs Income Level}
    \end{minipage}\hfill
    \begin{minipage}{0.45\textwidth}
        \centering
        \includegraphics[width=0.9\textwidth]{social_levels} % first figure itself
        \caption{Social Spending vs Income Level}
    \end{minipage}\hfill
    \begin{minipage}{0.45\textwidth}
        \centering
        \includegraphics[width=0.9\textwidth]{fixed_capital_levels} % first figure itself
        \caption{Infrastructure Investment vs  Income Level}
    \end{minipage}\hfill
    \begin{minipage}{0.45\textwidth}
        \centering
        \includegraphics[width=0.9\textwidth]{loan_levels} % first figure itself
        \caption{Personal Loans vs Income Level}
    \end{minipage}\hfill
    \begin{minipage}{0.45\textwidth}
        \centering
        \includegraphics[width=0.9\textwidth]{literacy_levels} % first figure itself
        \caption{Literacy vs Income Level}
    \end{minipage}\hfill
    \begin{minipage}{0.45\textwidth}
        \centering
        \includegraphics[width=0.9\textwidth]{elec_levels} % first figure itself
        \caption{Electricity vs Income Level}
    \end{minipage}\hfill
    \begin{minipage}{0.45\textwidth}
        \centering
        \includegraphics[width=0.9\textwidth]{imr_levels} % first figure itself
        \caption{Infant Mortality vs Income Level}
    \end{minipage}\hfill
    \begin{minipage}{0.45\textwidth}
        \centering
        \includegraphics[width=0.9\textwidth]{agshare_levels} % second figure itself
        \caption{Share of Agr. vs Income Level}
    \end{minipage}
\end{figure}
\clearpage
\par
	The next estimation explores the possibility of states having their own steady-state income levels, formally testing the link between income growth and production fundamentals.  Table 2 displays first the OLS regression of a fixed effects panel data model, then the GLS procedure for panel data due to the presence of heteroscedasticity mentioned earlier.\footnote{The estimated intercepts of each state are omitted to save space, but a linear regression of just state intercepts on growth resulted in a very low R$^{2}$, meaning that unobserved heterogeneity does not make up most of the higher R$^{2}$ in the conditional convergence models.}  The GLS regression should be more efficient than the OLS findings because of its asymptotic efficiency.  In all of the parameter estimates, the standard errors of the GLS specification are smaller than the OLS standard errors clustered at the state level.  Negative coefficients promote income convergence, while positive coefficients promote divergence.  The R$^{2}$ is still quite low, so the model still cannot wholly explain the variance in state-wise growth rates.\par

\begin{table}[!htbp] \centering 
  \caption{Fixed Effects Regression} 
  \label{Fixed Effects Regression} 
\begin{tabular}{@{\extracolsep{5pt}}lcc} 
\\[-1.8ex]\hline 
\hline \\[-1.8ex] 
 & \multicolumn{2}{c}{\textit{Dependent Variable: Growth of per capita SDP}} \\ 
 & OLS (1) & GLS (2) \\ 
\hline \\[-1.8ex]  
 log(SDP per capita) & $-$0.046$^{**}$ (0.023) & $-$0.048$^{***}$ (0.007) \\ 
  Population Growth & $-$1.032 (2.241) & $-$3.263 (1.695) \\ 
  log(Social Expenditure) & 0.014 (0.023) & 0.017$^{**}$ (0.006) \\ 
  log(Gross Fixed Capital Formation) & 0.016$^{***}$ (0.006) & 0.018$^{***}$ (0.001) \\ 
  log(Personal Loans) & 0.020$^{**}$ (0.009) & 0.021$^{***}$ (0.003) \\ 
  Literacy Rate & 0.001 (0.004) & 0.001 (0.001) \\ 
  Electricity Consumption per capita & 0.00004$^{*}$ (0.00002) & 0.00004$^{***}$ (0.000007) \\ 
  Infant Mortality Rate & $-$0.002$^{*}$ (0.001) & $-$0.001$^{***}$ (0.0003) \\ 
  Percentage Ag. Share SDP & $-$0.208 (0.161) & $-$0.253$^{***}$ (0.043) \\ 
 \hline \\[-1.8ex] 
Observations & 208 & 208 \\ 
R$^{2}$ & 0.340 & 0.361 \\ 
Adjusted R$^{2}$ & 0.258 & N/A \\ 
\hline 
\hline \\[-1.8ex] 
\textit{Note: OLS robust standard errors in parentheses.}  & \multicolumn{2}{r}{$^{*}$p$<$0.1; $^{**}$p$<$0.05; $^{***}$p$<$0.01} \\ 
\end{tabular} 
\end{table} 

The log of SDP per capita is found to be significantly negative at the 5\% significance level in the OLS specification and at the 10\% significance level in the GLS specification, but the parameter estimate is quite small.  This confirms evidence of conditional convergence across Indian states at a rate of about 4-5\% a year, albeit to stratified steady states.\footnote{The neoclassical model predicts a convergence speed of 5-6\% \citep{barro_convergence_1992}.}  Population growth has a negative but insignificant effect on growth in both specifications, providing only weak evidence to the Malthusian argument.  The standard error was greatly attenuated in the GLS model, but the parameter is still statistically insignificant, even at the 10\% significance level.  The statistical relationship in this model is ambiguous, but high fertility rates are typically symptomatic of poverty.  Social expenditure as a percentage of SDP is weakly positive then significantly positive in the OLS and GLS estimations respectively.  This confirms the belief that a state’s social expenditure can have a positive impact on growth and narrowing income gaps.  Gross fixed capital formation as a percentage of SDP is small but positive and statistically significant at the 1\% significance level in both the OLS and GLS estimations.  This underscores the importance of infrastructure investment in promoting growth.  Personal loans as a percentage of SDP is significantly positive in both specifications, but it is significant at the 1\% significance level in the GLS estimation.  This parameter supports the notion that a state’s financial infrastructure promotes growth.  This is particularly relevant in India where the vast majority of the population faces severe credit constraints, particularly due to the prevalence of informal salaries and a lack of income proof.  States can do more to promote their financial ecosystems, but the long term impact of increased indebtedness is a potential problem down the road.  The literacy rate has surprisingly little bearing in both specifications, but some of its impact could be captured in the state fixed effects.  The intuition of the positive influence of human capital in growth theory is well-documented, and statistically insignificant findings should not deter investment in education and the like.\footnote{The unambiguous positive effect that education has on quality of life is also important to consider in the context of welfare and inequality.}  Electricity consumption per capita coefficient is positive and significant in both estimations, but is very small, suggesting a limited influence on economic growth.  The purpose of its inclusion was to attempt to separate infrastructure investment with infrastructure stock, to try and isolate infrastructure quality and efficiency from simply spending, but it is likely that a large part of its influence was captured by the coefficient of gross fixed capital formation.  The infant mortality rate was significantly negative in both models, although quite small.  This is consistent with the understanding that lower population health impedes growth because of lower labor productivity.  The percentage share of agriculture of SDP has a negative coefficient which is insignificant in the OLS regression, but significant in the GLS regression thanks to the reduced standard error.  This supports the finding that a higher share of agriculture in SDP acts as a drag on growth due the decreasing returns to scale inherent to it.  Farmers in India also rely heavily on government support, particularly in the case of a poor monsoon season, and can create a serious drag on government resources as infrastructure essential to agriculture such as irrigation and rural electrification is purveyed by state governments \citep{ahluwalia_state-level_2000}.  States such as Maharashtra, Tamil Nadu and Gujarat have benefited from an economy oriented more towards services and manufacturing -- together they produce about 40\% of industrial and service sector output \citep{purfield_mind_2006}.\par

\begin{table}[!htbp] \centering 
  \caption{Fixed Effects Regression w/ Time-Fixed Effects} 
  \label{} 
\begin{tabular}{@{\extracolsep{5pt}}lcc} 
\\[-1.8ex]\hline 
\hline \\[-1.8ex] 
 & \multicolumn{2}{c}{\textit{Growth of per capita SDP}} \\ 
 & OLS (3) & GLS (4) \\ 
\hline \\[-1.8ex]
 log(SDP per capita) & $-$0.271$^{***}$ (0.066) & $-$0.289$^{***}$ (0.024) \\  
  Population Growth & $-$0.603 (2.274) & $-$1.211 (2.191) \\ 
  log(Social Expenditure) & 0.031 (0.023) & 0.036$^{***}$ (0.011) \\ 
  log(Gross Fixed Capital Formation) & 0.007$^{*}$ (0.004) & 0.010$^{***}$ (0.003) \\ 
  log(Personal Loans) & $-$0.013 (0.021) & $-$0.0096 (0.0095) \\ 
  Literacy Rate & $-$0.002 (0.002) & $-$0.004$^{**}$ (0.001) \\ 
  Electricity Consumption per capita & 0.0001 (0.00004) & 0.00007$^{***}$ (0.00001) \\ 
  Infant Mortality Rate & $-$0.001 (0.001) & $-$0.0002 (0.0005) \\ 
  Percentage Ag. Share SDP & $-$0.351$^{**}$ (0.170) & $-$0.340$^{***}$ (0.075) \\ 
 \hline \\[-1.8ex] 
Observations & 208 & 208 \\ 
R$^{2}$ & 0.610 & 0.623 \\ 
Adjusted R$^{2}$ & 0.528 & N/A \\ 
\hline 
\hline \\[-1.8ex] 
\textit{Note: OLS robust standard errors in parentheses.}  & \multicolumn{2}{r}{$^{*}$p$<$0.1; $^{**}$p$<$0.05; $^{***}$p$<$0.01} \\ 
\end{tabular} 
\end{table} 

In Table 3, time fixed effects were included after conducting an F test for their joint significance.  The year-wise dummy estimates were omitted in the interest of succinctness.  The R$^{2}$ nearly doubled from about 30\% to 60\% of the variance identified, reflecting the significance of time-variant heterogeneity across years.  The explanatory power of lagged SDP per capita on reducing income growth increased immensely, and it is statistically significant at the 1\% significance level in both OLS and GLS estimations.  The estimated convergence rate becomes nearly 30\% a year, although this is likely due to serious upward bias from using fixed effects, as outlined in Shioji (\citeyear{shioji1997convergence}).  Shioji finds that OLS regressions with fixed effects are subject to measurement error biases that not trivial, but huge.  He shows that the estimated speed of convergence from the OLS with fixed effects is biased upwards by as much as 7-15\%.  It is quite possible that the 4-5\% rate in the previous regression is also the result of upward bias.  The magnitude of the population growth estimate decreased, but it is still negative and statistically insignificant in both methods.  The coefficient of social expenditure grew and is significant in the GLS estimate at a 1\% significance level.  Gross fixed capital formation has a smaller coefficient, but is still significantly positive.  Strangely, the sign of the coefficient on personal loans changed to negative, but the finding is statistically insignificant, and contains zero in the 95\% confidence interval.  Literacy rate is still negative and insignificant in the OLS regression, but becomes significantly negative at the 5\% significance level in the GLS estimation.  This could be due to a number of reasons, from sampling error caused by inflated figures or cheating by teachers, or the inability of literacy to counteract a general lack of economic opportunity.  Purfield (\citeyear{purfield_mind_2006}) remarks that, although India is a labor-rich country, growth in India has not been job-intensive.\footnote{The national growth-employment elasticity for the organized sector is only 0.5, the elasticity for the unorganized sector is immeasurable.}  After all, human skills are only labor augmenting factors of production, and cannot independently promote growth.  Electricity consumption per capita is still small and positive, but only significant in the GLS regression.  Infant mortality rate is negative, but became statistically insignificant, suggesting that time fixed effects could be capturing a lot of important human capital effects that are not measured by estimates of literacy or infant mortality.  Percentage share of agriculture in SDP is still negative, but larger and significant in both specifications.  \par
Evidence of $\sigma$ divergence and a lack of evidence of unconditional $\beta$ divergence suggest that inequality across states in India is unequivocally increasing.  Evidence of conditional $\beta$ convergence indicates that this growing disparity is linked to inter-state differences in steady-state income potential.  The wide gaps in these steady-state incomes are correlated with variations in human capital, economic structure, public investment and financial infrastructure.  This clearly indicates that $\beta$ convergence is not sufficient for $\sigma$ convergence because the steady-state levels of income may diverge over time.  While some of the insignificant parameters are surprising, a common belief is that insignificance could reflect a variable’s multicollinearity with other effects captured in the fixed effects, dampening its own independent influence \citep{datt_why_1998,rao_convergence_1999,ahluwalia_state-level_2000}.  This could explain the particularly counterintuitive findings on the proxies for human capital.\par

\section{Conclusion}
Despite the sometimes questionable theory and assumptions, the results are mostly intuitive.  Although the magnitudes of the findings are questionable, given the bias they likely suffer (see Shioji (\citeyear{shioji1997convergence})), the coefficients are largely as expected and point to the importance of production fundamentals in driving growth.  The evidence of conditional convergence in particular points to the possibility that their unequal factor endowments of states constrains them to distinct and uneven steady-state income levels, eliminating the possibility of unconditional convergence to a common steady-state income.  \par   
Whether it was complacency or inability, the Indian Government failed to eliminate this endowment inequality before the process of development was largely taken over by the market.  States which did not benefit from reforms must be assisted through addressing the specific productivity deficiencies holding them back.  Although social expenditure and education have little bearing in this specification, a litany of studies supports their role in equitable and sustainable growth.  The process of state competition, where states compete to create business-friendly environments and attract investment, has palpable efficiency gains, but risks leaving poor states behind at the behest of corrupt chief ministers.  India's democracy is growing more fissiparous over time as its states' politics are increasingly characterized by regionalization and populism, and it seems unlikely that the changes that need to made will occur naturally \citep{ahluwalia_state-level_2000}.  Barring a substantial amount of reform and rationalization of business, states like Bihar are unlikely to attract large amounts of capital in the near future.  Hence, the Central Government must step back in to address these inequalities, or incentivize the private sector to do so, given their previous failure.  This could take the form of Public-Private Partnerships, or somehow aligning the incentives of the chief ministers of the poorer states with their own people.  If market forces are left unfettered, wealth may continue to concentrate in some of the richest states and cities, bringing more and more productivity advantages as outlined in Myrdal's cumulative causation model.  It is quite likely that, in the near future, evidence of unconditional divergence may be observed instead.


\clearpage
\bibliography{Dissertation.bib}
\nocite{*}
\section*{Data Appendix}
Data for this study was primarily gathered from the Reserve Bank of India's (RBI) Handbook of Statistics on Indian States,\footnote{\linkurl{https://rbi.org.in/home.aspx}.} and supplemented with data from \linkurl{www.indiastat.com}.  
\begin{itemize}
\item SDP data was taken from the RBI's Gross State Domestic Product at Factor Cost (Constant Prices) spreadsheet, but was deflated to 1980-81 prices.
\item Population was taken from the RBI's State-wise Total Population spreadsheet, but was interpolated using approx() in RStudio due to there only be decennial values coinciding with the Indian Census.  These were used to create interpretable per capita values, namely income per capita.
\item Social Expenditure was taken from the RBI's State-wise Social Sector Expenditure spreadsheet, and was taken as a percentage of SDP.
\item Gross Fixed Capital Formation was taken from the RBI's State-wise Gross Fixed Capital Formation spreadsheet, and was taken as a percentage of SDP.
\item Personal Loans were taken from the RBI's State-wise Personal Loans by Scheduled Commercial Banks spreadsheet, and was taken as a percentage of SDP.
\item Infant Mortality Rate was taken from the RBI's State-wise Infant Mortality Rate spreadsheet, but regrettably data was limited to 1997 onwards.
\item Percentage Agriculture Share of SDP was created using the RBI's Sectoral Gross State Domestic Product at Factor Cost: Agriculture (Constant Prices) spreadsheet.  
\item Electricity Consumption per Capita was taken from \linkurl{www.indiastat.com}, 
\item Income ranks and growth rates were calculated in RStudio by the author.
\end{itemize}

\textbf{Word Count}: 9,911

\end{document}
